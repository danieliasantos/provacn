\documentclass[12pt]{article}

\usepackage{answers}
\usepackage{setspace}
\usepackage{graphicx}
\usepackage{enumitem}
\usepackage{multicol}
\usepackage{mathrsfs}
\usepackage[margin=1in]{geometry} 
\usepackage{amsmath,amsthm,amssymb}
\usepackage[brazil]{babel}  
\usepackage[utf8]{inputenc}
\usepackage{algpseudocode,algorithm}	% pseudo-codigo
 
\newcommand{\N}{\mathbb{N}}
\newcommand{\Z}{\mathbb{Z}}
\newcommand{\C}{\mathbb{C}}
\newcommand{\R}{\mathbb{R}}

\DeclareMathOperator{\sech}{sech}
\DeclareMathOperator{\csch}{csch}
 
\newenvironment{theorem}[2][Teorema]{\begin{trivlist}
\item[\hskip \labelsep {\bfseries #1}\hskip \labelsep {\bfseries #2.}]}{\end{trivlist}}
\newenvironment{definition}[2][Definição]{\begin{trivlist}
\item[\hskip \labelsep {\bfseries #1}\hskip \labelsep {\bfseries #2.}]}{\end{trivlist}}
\newenvironment{proposition}[2][Proposição]{\begin{trivlist}
\item[\hskip \labelsep {\bfseries #1}\hskip \labelsep {\bfseries #2.}]}{\end{trivlist}}
\newenvironment{lemma}[2][Lema]{\begin{trivlist}
\item[\hskip \labelsep {\bfseries #1}\hskip \labelsep {\bfseries #2.}]}{\end{trivlist}}
\newenvironment{exercise}[2][Exercício]{\begin{trivlist}
\item[\hskip \labelsep {\bfseries #1}\hskip \labelsep {\bfseries #2.}]}{\end{trivlist}}
\newenvironment{solution}[2][Solução]{\begin{trivlist}
\item[\hskip \labelsep {\bfseries #1}]}{\end{trivlist}}
\newenvironment{problem}[2][Problema]{\begin{trivlist}
\item[\hskip \labelsep {\bfseries #1}\hskip \labelsep {\bfseries #2.}]}{\end{trivlist}}
\newenvironment{question}[2][Questão]{\begin{trivlist}
\item[\hskip \labelsep {\bfseries #1}\hskip \labelsep {\bfseries #2.}]}{\end{trivlist}}
\newenvironment{corollary}[2][Corolário]{\begin{trivlist}
\item[\hskip \labelsep {\bfseries #1}\hskip \labelsep {\bfseries #2.}]}{\end{trivlist}}
 
% Declaracoes
\algrenewcommand\algorithmicend{\textbf{fim}}
\algrenewcommand\algorithmicdo{\textbf{faça}}
\algrenewcommand\algorithmicwhile{\textbf{enquanto}}
\algrenewcommand\algorithmicfor{\textbf{para}}
\algrenewcommand\algorithmicif{\textbf{se}}
\algrenewcommand\algorithmicthen{\textbf{então}}
\algrenewcommand\algorithmicelse{\textbf{senão}}
\algrenewcommand\algorithmicreturn{\textbf{devolve}}
\algrenewcommand\algorithmicfunction{\textbf{função}}

\algrenewtext{EndWhile}{\algorithmicend\ \algorithmicwhile}
\algrenewtext{EndFor}{\algorithmicend\ \algorithmicfor}
\algrenewtext{EndIf}{\algorithmicend\ \algorithmicif}
\algrenewtext{EndFunction}{\algorithmicend\ \algorithmicfunction}

\algnewcommand\algorithmicto{\textbf{até}}
\algrenewtext{For}[3]%
{\algorithmicfor\ #1 $\gets$ #2 \algorithmicto\ #3 \algorithmicdo}
 
\begin{document}
 
% --------------------------------------------------------------
%                         Start here
% --------------------------------------------------------------
 
\title{Relatório com soluções da Segunda Prova}%replace with the appropriate homework number
\author{Daniel Elias e Matheus Saliba\\ %replace with your name
Cálculo Numérico\\
2018.01} %if necessary, replace with your course title
 
\maketitle
\newpage
\begin{question}{1}
	Para solução da questão, primeiramente foi calculado \textit{f(1,7)}, utilizando a função $f(x) = \sqrt[3]{(x - 1)}$, onde \textit{pX} era o ponto a ser interpolado, e \textit{pY} o valor exato da função, como a seguir:\\
	\begin{algorithm}
		\begin{algorithmic}[1]
			\State{$pX \gets 1.7$})
			\State{$pY \gets (pX - 1) ^ {\frac{1}{3}}$} \Comment{$pY = 0.88790$}
		\end{algorithmic}
	\end{algorithm} \\
	Após, foi calculada a Interpolação Linear, de grau um (\textit{P}\textsubscript{1}), utilizando a expressão $p1 = (\frac{y(3) - y(2)}{x(3) - x(2)}) \times (pX - x(2))$, utilizando os pontos \textit{1,6} e \textit{1,9}, mais próximos do ponto buscado. Resultado: $\textit{P}\textsubscript{1} = 0.88412$. Foi calculado, também, o erro absoluto em relação à função original: $erroP1 = p1 - pY$, tendo por resultado, $-0.0037858$.\\
	
	Na sequencia foi calculada a Interpolação Quadrática, de grau dois (\textit{P}\textsubscript{2}), utilizando a expressão: \\
	
	$p2 = y(1) \times \frac{(pX - x(2)) \times (pX - x(3))}{(x(1) - x(2)) \times (x(1) - x(3))} + y(2) \times \frac{(pX - x(1)) \times (pX - x(3))}{(x(2) - x(1)) \times (x(2) - x(3))} + y(3) \times \frac{(pX - x(1)) \times (1.7 - x(2))}{(x(3) - x(1)) \times (x(3) - x(2))} $ \\
	
	Nesta expressão, todos os pontos dados foram utilizados. Resultado: $\textit{P}\textsubscript{2} = 0.88864$. O erro absoluto calculado foi: $erroP2 = p2 - pY = 7.3752e-04$. \\
	
	A seguir, foi calculada a Interpolação Cúbica, de grau três (\textit{P}\textsubscript{3}), utilizando a função interna do Octave \textit{interp1}, como a seguir: \\
	
	$p3 = interp1(x, y, pX, 'cubic')$ \\
	
	Neste caso, o resultado foi: $\textit{P}\textsubscript{3} = 0.88777$. O erro absoluto foi: $erroP3 = p3 - pY = -1.3147e-04$. A seguir, as funções foram plotadas para comparação. Como esperado, a curva de interpolação que mais se aproximou da curva da função foi a Interpolação Cúbica $\textit{P}\textsubscript{3}$.

\end{question}

\newpage
\begin{question}{2}
	Utilizando \textit{x} como o eixo tempo em dias e \textit{y1} como amostra 1 e \textit{y2} como amostra 2, a modelagem ocorreu de acordo com o enunciado, ficando assim:\\
	$x = [0,6,10,13,17,20,28];$\\
	$y1 = [6.6700,17.3300,42.6700,37.3300,30.1000,29.3100,28.7400];$\\
	$y2 = [6.6700,16.1100,18.8900,15.0000,10.5600,9.4400,8.8900];$\\

	Foi utilizado também um domínio \textit{a}, que vai de 0 a 28 com passo de 0,01: $a \gets [0:.01:28];$. Logo em seguida, uma variável de controle \textit{n} é definida como tamanho do vetor a para controle do laço que manipula a criação da curva em que são chamados os elementos do enunciado na função \textit{InterpolacaoLagrange()}:\\
	\begin{algorithm}
		\begin{algorithmic}[1]
			\State{$n \gets size(a,2);$}
			\For{i}{1}{n}
			\State {$curva[i] \gets InterpolacaoLagrange(x, y1, a(i))$} \Comment{Preenche o vetor}
			\EndFor
		\end{algorithmic}
	\end{algorithm}\\

	Então, começamos a fazer o \textit{plot} da amostra 1 com uma curva em vermelho, utilizando o vetor \textit{a} como domínio e logo em seguida descobrindo os valores mínimo e máximo do vetor, assim como pede a questão B:\\
	
	\begin{algorithm}
		\begin{algorithmic}[1]
			\State{$plot(a,curva1,'r'); hold on$}
			\State{$min1 = min(curva1)$}
			\State{$max1 = max(curva1)$}
		\end{algorithmic}
	\end{algorithm}

	O exato procedimento se repete para a amostra 2, onde a curva é em azul:\\
	\begin{algorithm}
		\begin{algorithmic}[1]
			\For{i}{1}{n}
			\State {$curva[i] \gets InterpolacaoLagrange(x, y2, a(i))$}
			\EndFor
			\State{$plot(a,curva2,'b');$}
			\State{$min2 = min(curva2)$}
			\State{$max2 = max(curva2)$}
		\end{algorithmic}
	\end{algorithm}\\
	E, para finalizar, a declaração e ajustes dos elementos no gráfico como titulo, rótulos e legendas.\\

\end{question}

\newpage
\begin{question}{3}
	A questão 3 foi solucionada na seguinte forma:\\
	Na solução da letra A (Questa03A.m), foram utilizadas as bases, sendo que na ordenada foi informada a altura dos funcionários, e na abscissa o seu peso, e plotado o gráfico de dispersão.\\
	$vetorX = [183, 173, 168, 188, 158, 163, 193, 163, 178];$\\
	$vetorY = [79, 69, 70, 81, 61, 63, 79, 71, 73];$\\
	
	Para resolução da letra B (Questa03B.m), foi utilizado o algoritmo de ajuste pelos mínimos quadrados, tendo como resultado o $b \textsubscript{0} = -20.078$ e $b \textsubscript{1} = 0.52757$, utilizando a altura em centímetros. O \textit{plot} do gráfico contém o Diagrama de Dispersão, da letra A, e a reta de ajuste, para comparação\\
	
	A resolução da letra C (Questa03C.m) utilizou o algoritmo de Interpolação Cúbica para estimar o peso e altura, conforme enunciado. Os resultados foram: peso de 70.379kg para altura de 175cm e 191.44cm para peso de 80kg. Neste caso, os dados foram ordenados do enunciado foram ordenados, e utilizados somente os que eram mais próximos ao solicitado:\\
	
	Estimativa de peso para altura de 175cm:\\
	\begin{algorithm}
		\begin{algorithmic}[1]
			\State{$vetorX = [173, 178, 183];$}
			\State{$vetorY = [69, 73, 79];$}
			\State{$a = 175;$}
			\State{$fA = interp1(vetorX, vetorY, a, 'cubic')$}
			\State{$fA = 70.379$}
		\end{algorithmic}
	\end{algorithm}
	
	Estimativa de altura para peso de 80kg:\\
	\begin{algorithm}
		\begin{algorithmic}[1]
			\State{$vetorX = [73, 79, 81];$}
			\State{$vetorY = [178, 193, 188];$}
			\State{$p = 80;$}
			\State{$fP = interp1(vetorX, vetorY, p, 'cubic')$}
			\State{$fP = 191.44$}
		\end{algorithmic}
	\end{algorithm}
	
	A letra D (Questa03D.m) foi resolvida como a letra B, mas invertendo-se os dados: a ordenada recebeu o peso e a abscissa recebeu a altura. Foi utilizado, também, o algoritmo de ajuste por mínimos quadrados, tendo como resultado o $b \textsubscript{0} = 60.295$ e $b \textsubscript{1} = 1.5857$, utilizando a altura em centímetros. O \textit{plot} do gráfico contém o Diagrama de Dispersão e a reta de ajuste, para comparação.\\
	
	Não houve entendimento sobre como resolver a letra E.
	
\end{question}

\newpage
\begin{question}{4}
	A modelagem foi feita de acordo com o seguinte:\\
	
	Na letra A (Arquivo Questao4a.m), o algoritmo (spline3.m) utiliza uma matriz de coordenadas correspondentes, logo, de acordo com o enunciado a declaração de X ficou assim:
\\
	
	$X = [0.9, 1.3, 1.9, 2.1, 2.6, 3.0, 3.9, 4.4, 4.7, 5, 6, 7, 8, 9.2, 10.5, 11.3, 11.6, 12, 12.6, 13, 13.3; \\1.3, 1.5, 1.85, 2.1, 2.6, 2.7, 2.4, 2.15, 2.05, 2.1, 2.25, 2.3, 2.25, 1.95, 1.4, 0.9, 0.7, 0.6, 0.5, 0.4, 0.25];
$ \\
	
	Onde a primeira coluna é o x e a segunda coluna é o respectivo $f(x)$. Utilizando então a variável X é feita a chamada do código spline3.m:\\
	
	$Spline = spline3(X);$\\
	
	No próprio código do spline3.m é feito o plot do pato e com asteriscos(*) são marcados os pontos originais de X, para efeito de comparação.\\
	
	Na letra B (Arquivo Questao4B.m), vamos utilizar novamente o InterpolacaoLagrange.m, então precisamos declarar dois vetores e um dominio com n=20, assim como pedido no enunciado:\\
	
	$x = [0.9, 1.3, 1.9, 2.1, 2.6, 3.0, 3.9, 4.4, 4.7, 5, 6, 7, 8, 9.2, 10.5, 11.3, 11.6, 12, 12.6, 13, 13.3];
$\\
	$y = [1.3, 1.5, 1.85, 2.1, 2.6, 2.7, 2.4, 2.15, 2.05, 2.1, 2.25, 2.3, 2.25, 1.95, 1.4, 0.9, 0.7, 0.6, 0.5, 0.4, 0.25];
$\\
	$dom = linspace(0.9,13.3,20);
$\\
	
	E então a função é chamada:
\\
	
	$Lagrange = InterpolacaoLagrange(x,y,dom)
$\\
	
	Agora, com os resultados do Spline e Lagrange, o arquivo "Questao4.m" utiliza o mesmo domínio de n=20 para plotar Lagrange ao lado do spline e após fazer a declaração e formatação do gráfico:\\

	\begin{algorithm}
		\begin{algorithmic}[1]
			\State{$[ Spline ] = Questao4A();$}
			\State{$[ Lagrange ] = Questao4B();$}
			\State{$dom = linspace(0.9,13.3,20);$}
			\State{$plot(dom,Lagrange,'k');$}
			\State{$title("Questao 4");$}
			\State{$h = legend ("Spline", "Lagrange");$}
			\State{$legend (h, "location", "northeastoutside");$}
			\State{$set(h, "fontsize", 20);$}
			\State{$hold on;$}
		\end{algorithmic}
	\end{algorithm}
		
\end{question}

\end{document}
