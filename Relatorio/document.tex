\documentclass[12pt]{article}

\usepackage{answers}
\usepackage{setspace}
\usepackage{graphicx}
\usepackage{enumitem}
\usepackage{multicol}
\usepackage{mathrsfs}
\usepackage[margin=1in]{geometry} 
\usepackage{amsmath,amsthm,amssymb}
\usepackage[brazil]{babel}  
\usepackage[utf8]{inputenc}
 
\newcommand{\N}{\mathbb{N}}
\newcommand{\Z}{\mathbb{Z}}
\newcommand{\C}{\mathbb{C}}
\newcommand{\R}{\mathbb{R}}

\DeclareMathOperator{\sech}{sech}
\DeclareMathOperator{\csch}{csch}
 
\newenvironment{theorem}[2][Teorema]{\begin{trivlist}
\item[\hskip \labelsep {\bfseries #1}\hskip \labelsep {\bfseries #2.}]}{\end{trivlist}}
\newenvironment{definition}[2][Definição]{\begin{trivlist}
\item[\hskip \labelsep {\bfseries #1}\hskip \labelsep {\bfseries #2.}]}{\end{trivlist}}
\newenvironment{proposition}[2][Proposição]{\begin{trivlist}
\item[\hskip \labelsep {\bfseries #1}\hskip \labelsep {\bfseries #2.}]}{\end{trivlist}}
\newenvironment{lemma}[2][Lema]{\begin{trivlist}
\item[\hskip \labelsep {\bfseries #1}\hskip \labelsep {\bfseries #2.}]}{\end{trivlist}}
\newenvironment{exercise}[2][Exercício]{\begin{trivlist}
\item[\hskip \labelsep {\bfseries #1}\hskip \labelsep {\bfseries #2.}]}{\end{trivlist}}
\newenvironment{solution}[2][Solução]{\begin{trivlist}
\item[\hskip \labelsep {\bfseries #1}]}{\end{trivlist}}
\newenvironment{problem}[2][Problema]{\begin{trivlist}
\item[\hskip \labelsep {\bfseries #1}\hskip \labelsep {\bfseries #2.}]}{\end{trivlist}}
\newenvironment{question}[2][Questão]{\begin{trivlist}
\item[\hskip \labelsep {\bfseries #1}\hskip \labelsep {\bfseries #2.}]}{\end{trivlist}}
\newenvironment{corollary}[2][Corolário]{\begin{trivlist}
\item[\hskip \labelsep {\bfseries #1}\hskip \labelsep {\bfseries #2.}]}{\end{trivlist}}
 
\begin{document}
 
% --------------------------------------------------------------
%                         Start here
% --------------------------------------------------------------
 
\title{Relatório da Segunda Prova}%replace with the appropriate homework number
\author{Daniel Elias e Matheus Saliba\\ %replace with your name
Cálculo Numérico\\
2018.01} %if necessary, replace with your course title
 
\maketitle
%Below is an example of the problem environment
\begin{question}{6}
\begin{enumerate}[label=\alph*)]
    \item Suppose an entire function $f$ is bounded by $M$ along $|z|=R$. Show that the coefficients $C_k$ in its power series expansion about $0$ satisfy
    \[
    |C_k|\leq\frac{M}{R^k}.
    \]
    \item Suppose a polynomial is bounded by $1$ in the unit disc. Show that all its coefficients are bounded by 1.
\end{enumerate}
\end{question}

%Below is the solution environment
\begin{solution}{}
Part a): Since $f$ is an entire function it can be expressed as an infinite power series, i.e.
\[
f(z)=\sum_{k=0}^\infty\frac{f^{(k)}(0)}{k!}z^k=\sum_{k=0}^\infty C_kz^k.
\]
If we recall Cauchy's Integral we have
\[
f(z)=\frac{1}{2\pi i}\int_\gamma\frac{f(w)}{w-z}\ dw,
\]
carefully notice that $\frac{1}{w-z}=\frac{1}{w}\cdot\frac{1}{1-\frac{z}{w}}$ can be written as a geometric series. We have

%The align environment with no label
\begin{align*}
\frac{1}{2\pi i}\int_\gamma\frac{f(w)}{w-z}\ dw &=\frac{1}{2\pi i}\int_\gamma\left\lbrace\frac{f(w)}{w}\cdot\left(\frac{1}{1-\frac{z}{w}}\right) \right\rbrace\ dw\\[8pt]
&=\frac{1}{2\pi i}\int_\gamma\left\lbrace\frac{f(w)}{w}\cdot\left(1+\frac{z}{w}+\frac{z^2}{w^2}+\frac{z^3}{w^3}+\cdots\right) \right\rbrace\ dw\\[8pt]
&=\left(\frac{1}{2\pi i}\int_\gamma \frac{f(w)}{w}\ dw\right)z^0+\left(\frac{1}{2\pi i}\int_\gamma \frac{f(w)}{w^2}\ dw\right)z^1+\left(\frac{1}{2\pi i}\int_\gamma \frac{f(w)}{w^3}\ dw\right)z^2\cdots
\end{align*}
Now take the modulus of $C_k$ to get
\[
|C_k|=\left\lvert\frac{1}{2\pi i}\int_\gamma \frac{f(w)}{w^{k+1}}\ dw \right\rvert\leq\frac{1}{2\pi}\int_\gamma\frac{|f(w)|}{|w^{k+1}|}\ |dw|\leq \frac{M}{2\pi}\int_\gamma\frac{|dw|}{|w^{k+1}|}
\]
Then integrate along $\gamma(\theta)=Re^{i\theta}$ for $\theta\in [0,2\pi]$ to get
\[
|C_k|\leq \frac{M}{2\pi}\int_0^{2\pi}\frac{|iRe^{i\theta}\ d\theta|}{|R^{k+1}e^{ik\theta}|}=\frac{M}{2\pi\cdot R^k}\int_0^{2\pi}\ d\theta=\frac{M}{R^k}.
\]
Hence, $|C_k|\leq \frac{M}{R^k}$.
\end{solution}
\pagebreak

\end{document}
